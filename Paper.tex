%% 
%% Copyright 2007-2026 Elsevier Ltd
%% 
%% This file is part of the 'Elsarticle Bundle'.
%% ---------------------------------------------
%% 
%% It may be distributed under the conditions of the LaTeX Project Public
%% License, either version 1.3 of this license or (at your option) any
%% later version.  The latest version of this license is in
%%    http://www.latex-project.org/lppl.txt
%% and version 1.3 or later is part of all distributions of LaTeX
%% version 1999/12/01 or later.
%% 
%% The list of all files belonging to the 'Elsarticle Bundle' is
%% given in the file `manifest.txt'.
%% 
%% Template article for Elsevier's document class `elsarticle'
%% with numbered style bibliographic references
%% SP 2008/03/01
%% $Id: elsarticle-template-num.tex 289 2026-01-09 06:13:01Z rishi $
%%
 %%\documentclass[preprint,12pt]{elsarticle}

%% Use the option review to obtain double line spacing
%% \documentclass[authoryear,preprint,review,12pt]{elsarticle}

%% Use the options 1p,twocolumn; 3p; 3p,twocolumn; 5p; or 5p,twocolumn
%% for a journal layout:
%% \documentclass[final,1p,times]{elsarticle}
%%\documentclass[final,1p,times,twocolumn]{elsarticle}
%% \documentclass[final,3p,times]{elsarticle}
 %%\documentclass[final,3p,times,twocolumn]{elsarticle}
%% \documentclass[final,5p,times]{elsarticle}
 \documentclass[final,5p,times,twocolumn]{elsarticle}

%% For including figures, graphicx.sty has been loaded in
%% elsarticle.cls. If you prefer to use the old commands
%% please give \usepackage{epsfig}

%% The amssymb package provides various useful mathematical symbols
\usepackage{amssymb}
%% The amsmath package provides various useful equation environments.
\usepackage{amsmath}
%% The amsthm package provides extended theorem environments
%% \usepackage{amsthm}

%% The lineno packages adds line numbers. Start line numbering with
%% \begin{linenumbers}, end it with \end{linenumbers}. Or switch it on
%% for the whole article with \linenumbers.
%% \usepackage{lineno}

\journal{Nuclear Physics B}

\begin{document}

\begin{frontmatter}

%% Title, authors and addresses

%% use the tnoteref command within \title for footnotes;
%% use the tnotetext command for theassociated footnote;
%% use the fnref command within \author or \affiliation for footnotes;
%% use the fntext command for theassociated footnote;
%% use the corref command within \author for corresponding author footnotes;
%% use the cortext command for theassociated footnote;
%% use the ead command for the email address,
%% and the form \ead[url] for the home page:
%% \title{Title\tnoteref{label1}}
%% \tnotetext[label1]{}
%% \author{Name\corref{cor1}\fnref{label2}}
%% \ead{email address}
%% \ead[url]{home page}
%% \fntext[label2]{}
%% \cortext[cor1]{}
%% \affiliation{organization={},
%%             addressline={},
%%             city={},
%%             postcode={},
%%             state={},
%%             country={}}
%% \fntext[label3]{}

\title{Observer-Augmented Feedback Linearization for Residual Disturbance Rejection in Precision Two-Axis Gimbal Systems}

%% use optional labels to link authors explicitly to addresses:
%% \author[label1,label2]{}
%% \affiliation[label1]{organization={},
%%             addressline={},
%%             city={},
%%             postcode={},
%%             state={},
%%             country={}}
%%
%% \affiliation[label2]{organization={},
%%             addressline={},
%%             city={},
%%             postcode={},
%%             state={},
%%             country={}}

\author{Syed Shahid Mustafa} %% Author name

%% Author affiliation
\affiliation{organization={},%Department and Organization
            addressline={}, 
            city={},
            postcode={}, 
            state={},
            country={}}

%% Abstract
\begin{abstract}
Precision two-axis gimbals used in laser communication and optical tracking demand microradian-level pointing accuracy despite nonlinear dynamics, cross-axis coupling, and unmodeled disturbances such as friction, cable torque, and parameter uncertainty. Classical PID-based approaches are insufficient under these conditions. This paper presents an observer-augmented feedback linearization (FBL) control architecture for a 2-DOF gimbal system, where nominal nonlinear dynamics are canceled via computed torque control, and residual disturbances are compensated using a disturbance observer (DOB). The proposed approach preserves the simplicity of linear outer-loop control while significantly improving robustness against modeling errors. A high-fidelity simulation study demonstrates improved tracking accuracy and disturbance rejection compared to FBL-only control. Residual dynamics and practical limitations are analyzed, providing clear design guidelines for industrial gimbal systems.
\end{abstract}

%%Graphical abstract
\begin{graphicalabstract}
%\includegraphics{grabs}
\end{graphicalabstract}

%%Research highlights
\begin{highlights}
\item Research highlight 1
\item Research highlight 2
\end{highlights}

%% Keywords
\begin{keyword}
%% keywords here, in the form: keyword \sep keyword

%% PACS codes here, in the form: \PACS code \sep code

%% MSC codes here, in the form: \MSC code \sep code
%% or \MSC[2008] code \sep code (2000 is the default)

\end{keyword}

\end{frontmatter}

%% Add \usepackage{lineno} before \begin{document} and uncomment 
%% following line to enable line numbers
%% \linenumbers

\section{Introduction}
\begin{figure*}[!t] % [htbp] helps with placement (here, top, bottom, page)
	\centering
%	\includegraphics[width=6in]{block5.pdf} 
	\caption{Overall architecture of the proposed NDOB-augmented feedback linearization control strategy for the dual-stage gimbal system.}
	\label{Fig:block}
\end{figure*}
High-precision gimbaled optical terminals constitute a critical enabling technology for modern laser communication systems, including terrestrial free-space optical links, ground-to-space optical uplinks, and inter-platform laser communication \cite{dhruv2025review}. These systems impose exceptionally stringent line-of-sight (LOS) pointing requirements, often at the microradian or sub-microradian level, while operating in environments characterized by nonlinear dynamics, structural coupling, and external disturbances. Achieving such precision reliably remains a fundamental challenge in control system design \cite{dhruv2025review}.

A central difficulty arises from the inherent nonlinear and coupled dynamics of two-axis gimbal mechanisms. In practical implementations, the azimuth and elevation axes are dynamically coupled through inertia, Coriolis effects, and gravity-induced torques\cite{spong2006robot}. Classical linear control approaches, most notably PID-based architectures, are typically tuned around specific operating points and rely heavily on loop shaping to achieve acceptable performance \cite{dhruv2025review,dao2023nonlinear}. While such methods remain prevalent in industry due to their simplicity and robustness, their effectiveness deteriorates as the system operates over wide angular ranges or is subjected to significant disturbances and modeling uncertainties.

Feedback linearization (FBL), also referred to as computed torque control, provides a principled framework for addressing nonlinear multi-degree-of-freedom dynamics by explicitly canceling nominal nonlinearities using a model-based control law\cite{naderolasli2020two}. When the system model is exact, FBL transforms the nonlinear gimbal dynamics into decoupled linear subsystems, enabling the use of simple linear outer-loop controllers for trajectory tracking \cite{9279120}. This property makes FBL particularly attractive for precision pointing systems. However, in real-world gimbal systems, exact modeling is unattainable. Parameter uncertainty, unmodeled friction, cable harness effects, structural compliance, and actuator nonlinearities introduce residual dynamics that are not canceled by the nominal feedback linearization.

These residual dynamics become especially problematic in laser communication applications, where even small unmodeled torques can translate into significant LOS jitter\cite{kim2024system}. Moreover, terrestrial and ground-mounted laser communication gimbals are subject to base excitation caused by mast vibration, wind-induced motion, or structural oscillations\cite{li2024system}. In many practical scenarios, such base motion is not directly measurable with sufficient accuracy or bandwidth, rendering explicit feedforward compensation infeasible. As a result, the disturbance enters the gimbal dynamics implicitly and manifests as an effective torque disturbance acting on the joints \cite{dhruv2025review}.

Disturbance observer (DOB) techniques offer a compelling mechanism for addressing such residual effects. By estimating the discrepancy between the nominal model response and the actual system behavior, a DOB can reconstruct an equivalent disturbance torque and compensate for it in real time\cite{chen2015disturbance,tran2024observer}. DOBs have been successfully applied in motion control, robotics, and precision mechatronic systems to enhance robustness without sacrificing performance\cite{xu2022hybrid}. However, their integration with feedback linearization in high-accuracy gimbal systems—particularly under unmeasured base excitation—has not been sufficiently explored in the literature.

Existing studies on gimbal control often treat feedback linearization and disturbance rejection as separate design problems, or they assume that disturbances enter only as slowly varying torque biases\cite{naderolasli2020two}. In practice, base excitation violates these assumptions: it is dynamic, coupled, and enters the system through inertial effects rather than direct actuation channels. Without a careful observer-based formulation, the benefits of feedback linearization are severely compromised, and aggressive disturbance compensation can destabilize the closed-loop system or amplify sensor noise \cite{lv2023modeling}.

This motivates the development of an observer-augmented feedback linearization architecture that explicitly acknowledges the limitations of nominal model cancellation and systematically addresses residual dynamics through disturbance estimation. By embedding a disturbance observer within the computed torque framework, it becomes possible to preserve the structural advantages of feedback linearization while significantly improving robustness to unmodeled dynamics and unmeasured base excitation. Such an approach is particularly well suited to laser communication gimbals, where only joint torque actuation is available and high-fidelity sensing of base motion is often impractical.

%The objective of this work is therefore to provide a rigorous control framework for precision two-axis laser communication gimbals that (i) exploits feedback linearization to handle nonlinear and coupled dynamics, (ii) employs a disturbance observer to estimate and compensate residual torques arising from modeling errors and base excitation, and (iii) enables high-accuracy LOS stabilization without relying on direct measurement of base motion. By focusing on the interaction between feedback linearization and disturbance observation, this study aims to bridge an important gap between nonlinear control theory and the practical requirements of industrial laser communication systems. A simulation is developed which captures full fidelity of sensors behaviour and the external disturbances so that the study conform  to the near realistic behaviour of the model and controllers. 

The objective of this work is to provide a rigorous control framework for precision two-axis laser communication gimbals that: (i) exploits feedback linearization to decouple and linearize the nonlinear system dynamics; (ii) employs a nonlinear disturbance observer (NDOB) to estimate and compensate for residual torques arising from modeling uncertainties and base excitations; and (iii) enables high-accuracy line-of-sight (LOS) stabilization without requiring direct measurement of base motion. By focusing on the interaction between feedback linearization and disturbance observation, this study aims to bridge the gap between nonlinear control theory and the practical requirements of field-deployed laser communication systems. To validate the proposed approach, a high-fidelity numerical simulation is developed that incorporates realistic sensor noise and stochastic environmental disturbances, ensuring that the results conform to the practical constraints of industrial hardware.


%\begin{figure}[!t] % [htbp] helps with placement (here, top, bottom, page)
%	\centering
%\includegraphics[width=2.5in]{Fig_2_2.png} 
%\caption{A single column figure with six graphics.}
%	\label{FigGam}
%\end{figure}
%\begin{figure}[!t] % [htbp] helps with placement (here, top, bottom, page)
%	\centering
%	\includegraphics[width=2.5in]{Fig1_Gimbal.png} 
%	\caption{A single column figure with six graphics.}
%	\label{FigGam}
%\end{figure}







\section{Methodology}
\subsection{Gimbal System Dynamics}
The equations of motion for the two-degree-of-freedom (2-DOF)  are derived via the Euler-Lagrange formalism, resulting in the following nonlinear second-order system:  For a 2-DOF gimbal where the Elevation (inner) axis is nested within the Azimuth (outer) axis, the dynamics are characterized by significant nonlinear coupling \cite{spong2006robot}. The general equations of motion are expressed like the compact Manipulator Equation form:
\begin{equation}
	M(q)\ddot{q} + C(q,\dot{q})\dot{q} + G(q) + \mathbf{d} = \mathbf{\tau} 
	\label{eq:euler}
\end{equation}
Where:  $q = [\theta_{az}, \theta_{el}]^T$ is the state vector of joint angles.  $\mathbf{\tau} = [\tau_{az}, \tau_{el}]^T$ is the vector of motor torques.   $\mathbf{d}= [d_{az}, d_{el}]^T$ represents external disturbances (wind, mast vibration, and un-modeled dynamics of the gimbal system). The Mass (Inertia) Matrix $M(q)$ represents the configuration-dependent inertia. Because the elevation tilt changes the distribution of mass relative to the azimuth axis, $M$ is non-diagonal and nonlinear.
\begin{equation}
	M(q) = \begin{bmatrix} J_{az} + J_{el}\cos^2(\theta_{el}) & 0 \\ 0 & J_{el} \end{bmatrix} \nonumber
\end{equation}
Here $J_{az}$ is the  angular  inertia of the azimuth yoke.$J_{el}$ is the angular inertia of the elevation payload (laser optical assembly). As $\theta_{el} \to 90^\circ$ (Zenith), the effective azimuth inertia reduces to $J_{az}$. The Coriolis and Centripetal Matrix $C(q, \dot{q})$ captures the "fictitious" forces arising from the rotation of the nested frames. This is the "cross-coupling" term critical for precision tracking.
\begin{equation}
	C(q, \dot{q}) = \begin{bmatrix} -J_{el}\sin(2\theta_{el})\dot{\theta}_{el} & -J_{el}\sin(2\theta_{el})\dot{\theta}_{az} \\ \frac{1}{2}J_{el}\sin(2\theta_{el})\dot{\theta}_{az} & 0 \end{bmatrix} \nonumber
\end{equation}

The azimuth row, shows how elevation velocity affects azimuth torque (Coriolis). The elevation row  shows the "flinging" effect on the elevation axis caused by azimuth rotation (Centripetal). The Gravity Vector $G(q)$, for a gimbal that is not perfectly center-of-mass balanced (common in high-precision optical payloads), gravity exerts a position-dependent torque on the elevation axis.
\begin{equation}
	G(q) = \begin{bmatrix} 0 \\ m_{el} g r_{cg} \cos(\theta_{el}) \end{bmatrix} \nonumber
\end{equation}
$m_{el}$is the mass of the elevation payload.$r_{cg}$ is the  distance from the elevation pivot (axis) to the center of gravity.$g$ is the  gravitational acceleration ($9.81 \, \text{m/s}^2$).

%%The coupled-nonlinear system}
%We begin with  the nonlinear dynamics of the 2-DOF gimbal expressed in  the Euler–Lagrange framework \cite{spong2006robot}. The equations of motion are expressed in the standard manipulator form as given in  $\backslash${\tt{eq:euler}}
%The matrix $M(q) \in \mathbb{R}^{2\times2}$ is the symmetric positive-definite inertia matrix; notably, the azimuth inertia term $M_{11}$ varies with $\cos^{2}(\theta_{el})$, reflecting configuration-dependent inertial coupling. The matrix $C(q,\dot{q})$ represents Coriolis and centrifugal effects and captures the dynamic coupling that arises when both axes move simultaneously. The vector $G(q)$ accounts for gravitational torques. The term $\mathbf{d}$ denotes a lumped disturbance encompassing unmodeled dynamics, friction, wind-induced torques, and base excitation effects. The control input $\tau$ corresponds to the actuator torques applied at the gimbal joints.




\subsection{Feedback Linearization (Inner-Loop Control)}

The objective of feedback linearization is to cancel the nominal nonlinear dynamics such that the closed-loop system behaves as a set of decoupled double integrators,
\begin{equation}
\ddot{q} = \mathbf{v} \nonumber
\end{equation}
where $\mathbf{v}$ is a virtual control input \cite{spong2006robot}. Accordingly, the control torque is defined as
\begin{equation}
\tau = M(q)\mathbf{v} + C(q,\dot{q})\dot{q} + G(q) + \hat{\mathbf{d}} \nonumber
\end{equation}
where $\hat{\mathbf{d}}$ is the disturbance estimate provided by the nonlinear disturbance observer (NDOB). The virtual input $\mathbf{v}$ is designed in the outer loop using a linear control law.

\subsection{Nonlinear Disturbance Observer (NDOB) Design} To eliminate the steady-state tracking errors induced by unmodeled torques without the need for direct (and typically noise-corrupted) acceleration measurements, we implement a Nonlinear Disturbance Observer (NDOB). The objective is to estimate the lumped disturbance vector $\mathbf{d}$—which includes friction, cable harness torques, and parametric uncertainties—by utilizing the system’s measurable states and control inputs \cite{spong2006robot}.We introduce an auxiliary observer state $\mathbf{z}$ and a positive definite gain matrix $\mathbf{L} \in \mathbb{R}^{2\times2}$, typically configured as a diagonal matrix to decouple the azimuth and elevation estimation channels. To avoid the use of $\ddot{q}$, an auxiliary vector function $\mathbf{p}(q,\dot{q})$ is defined such that its derivative satisfies $\dot{\mathbf{p}} = \mathbf{L} \mathbf{M}(q)\ddot{q}$. For the 2-DOF gimbal system, an effective choice is:
\begin{equation}\mathbf{p}(q,\dot{q}) = \mathbf{L} \mathbf{M}(q)\dot{q}\nonumber
\end{equation}
The observer internal dynamics are then formulated as:
\begin{equation}\dot{\mathbf{z}} = -\mathbf{L}\mathbf{z} - \mathbf{L} \left[ \mathbf{C}(q,\dot{q})\dot{q} + \mathbf{G}(q) - \mathbf{\tau} + \mathbf{p}(q,\dot{q}) \right] \nonumber
\end{equation}

The reconstructed estimate of the lumped disturbance $\hat{\mathbf{d}}$ is recovered via:
\begin{equation}\hat{\mathbf{d}} = \mathbf{z} + \mathbf{p}(q,\dot{q})\nonumber
\end{equation}
By selecting $\mathbf{L}$ to be positive definite, the estimation error $\tilde{\mathbf{d}} = \mathbf{d} - \hat{\mathbf{d}}$ is guaranteed to converge exponentially to zero, provided the rate of change of the disturbance is slow relative to the observer dynamics \cite{chen2000nonlinear}.

\subsection{Augmented Feedback Linearization and Virtual Control Law} With the nonlinearities compensated by the nominal model and the residual disturbances mitigated by the NDOB, the control torque is defined in Eq. \eqref{eq:euler} 
%%\begin{equation}\mathbf{\tau} = \mathbf{M}(q)\mathbf{v} + \mathbf{C}(q,\dot{q})\dot{q} + \mathbf{G}(q) + \hat{\mathbf{d}}.\end{equation}
The virtual control input $\mathbf{v}$ is designed to enforce desired tracking characteristics using a Proportional-Derivative (PD) structure:

\begin{equation}\mathbf{v} = \ddot{q}_d + \mathbf{K}_p (q_d - q) + \mathbf{K}_d (\dot{q}_d - \dot{q})\nonumber
\end{equation}
where $\mathbf{K}_p$ and $\mathbf{K}_d$ are positive definite gain matrices. Defining the tracking error as $\mathbf{e} = q_d - q$, the closed-loop error dynamics become:

\begin{equation}\ddot{\mathbf{e}} + \mathbf{K}_d \dot{\mathbf{e}} + \mathbf{K}_p \mathbf{e} = \mathbf{M}^{-1}(q)(\mathbf{d} - \hat{\mathbf{d}}) \nonumber
\end{equation}
As $\hat{\mathbf{d}} \to \mathbf{d}$, the forcing term on the right-hand side asymptotically vanishes, reducing the system to an exponentially stable second-order linear differential equation \cite{chen2000nonlinear}.

\subsection{Simulation and Systematic Tuning Procedure} A bifurcated tuning strategy is adopted to ensure global stability and performance.
\begin{enumerate}
\item\textbf{ Nominal Tuning Phase:} The NDOB is initially deactivated ($\hat{\mathbf{d}} = 0$). The outer-loop gains $\mathbf{K}_p$ and $\mathbf{K}_d$ are tuned to achieve a critically damped response under nominal conditions. During this phase, the consistency of the feedback linearization is verified; specifically, the orientation-dependent inertia $\mathbf{M}(q)$ is monitored to ensure that cross-axis coupling—often manifested as azimuth oscillations during elevation slews—is effectively neutralized.

\item\textbf{	Robustness Augmentation Phase:} Upon stabilizing the nominal plant, the NDOB is activated. The observer gain $\mathbf{L}$ is incremented to broaden the disturbance rejection bandwidth. Low values of $\mathbf{L}$ result in sluggish compensation of friction, whereas excessively high gains may induce "peaking" or high-frequency chatter due to the amplification of sensor noise.
\end{enumerate}

The final tuned parameters ensure that modeling uncertainties, such as inaccurate moment-of-inertia calculations and center-of-gravity offsets, do not manifest as steady-state pointing errors, thereby enabling microradian-level precision.




\section{Stability and Convergence of the NDOB}To establish the theoretical validity of the proposed control architecture, we must demonstrate that the NDOB estimate $\hat{\mathbf{d}}$ asymptotically tracks the lumped disturbance $\mathbf{d}$. This convergence analysis is performed by examining the disturbance estimation error dynamics under the framework of Lyapunov stability theory \cite{chen2000nonlinear}.

\subsection{Disturbance Estimation Error Dynamics}Let the disturbance estimation error $\tilde{\mathbf{d}}$ be defined as:
\begin{equation}\tilde{\mathbf{d}} = \mathbf{d} - \hat{\mathbf{d}} \nonumber\end{equation}
Our objective is to prove that $\tilde{\mathbf{d}} \to 0$ as $t \to \infty$. In high-precision pointing applications, it is a standard and reasonable assumption that the lumped disturbance $\mathbf{d}$ (comprising friction and slowly varying cable torques) varies significantly slower than the observer's convergence rate, such that $\dot{\mathbf{d}} \approx \mathbf{0}$. Differentiating the estimation error with respect to time yields:
\begin{equation}\dot{\tilde{\mathbf{d}}} = \dot{\mathbf{d}} - \dot{\hat{\mathbf{d}}} \approx - \dot{\hat{\mathbf{d}}} \nonumber \end{equation}
Recalling the definition of the observer estimate $\hat{\mathbf{d}} = \mathbf{z} + \mathbf{p}(q, \dot{q}$, and noting that $\dot{\mathbf{p}} = \mathbf{L} \mathbf{M}(q) \ddot{q}$, the derivative of the estimate is:
\begin{equation}\dot{\tilde{\mathbf{d}}} = - \left( \dot{\mathbf{z}} + \mathbf{L} \mathbf{M}(q) \ddot{q} \right) \nonumber \end{equation}
Substituting the actual plant dynamics, $\mathbf{M}(q)\ddot{q} = \mathbf{\tau} - \mathbf{C}(q, \dot{q})\dot{q} - \mathbf{G}(q) - \mathbf{d}$, into the expression, and incorporating the observer internal state dynamics $\dot{\mathbf{z}}$, we obtain:
\begin{equation}\dot{\tilde{\mathbf{d}}} = - \left( -\mathbf{L}\mathbf{z} + \mathbf{L}[\mathbf{C}\dot{q} + \mathbf{G} - \mathbf{\tau} - \mathbf{p}] + \mathbf{L}[\mathbf{\tau} - \mathbf{C}\dot{q} - \mathbf{G} - \mathbf{d}] \right) \nonumber \end{equation}
By canceling the control torque $\mathbf{\tau}$ and the nominal nonlinear terms $\mathbf{C}\dot{q} + \mathbf{G}$, the expression simplifies to:
\begin{equation}\dot{\tilde{\mathbf{d}}} = \mathbf{L}(\mathbf{z} + \mathbf{p}) - \mathbf{L}\mathbf{d}\end{equation}
Substituting $\hat{\mathbf{d}} = \mathbf{z} + \mathbf{p}$ back into the equation, we arrive at the linear error dynamics:
\begin{equation}\dot{\tilde{\mathbf{d}}} = \mathbf{L}\hat{\mathbf{d}} - \mathbf{L}\mathbf{d} = -\mathbf{L}\tilde{\mathbf{d}} \nonumber
\end{equation}


\subsection{Lyapunov Stability Analysis}To verify the stability of the 2-DOF gimbal estimation loop, we propose a quadratic Lyapunov candidate function $V(\tilde{\mathbf{d}})$:\begin{equation}V(\tilde{\mathbf{d}}) = \frac{1}{2}\tilde{\mathbf{d}}^T \tilde{\mathbf{d}}\end{equation}The time derivative of the Lyapunov function along the error trajectories is given by:\begin{equation}\dot{V} = \tilde{\mathbf{d}}^T \dot{\tilde{\mathbf{d}}} = \tilde{\mathbf{d}}^T (-\mathbf{L}\tilde{\mathbf{d}})\end{equation}For $\dot{V}$ to be negative definite, the observer gain matrix $\mathbf{L}$ must be positive definite. In our implementation, $\mathbf{L}$ is selected as a diagonal matrix with strictly positive gains $\text{diag}(\lambda_{az}, \lambda_{el})$, ensuring that $\dot{V} < 0$ for all $\tilde{\mathbf{d}} \neq \mathbf{0}$. Consequently, the estimation error is exponentially stable, with a solution of the form $\tilde{\mathbf{d}}(t) = \tilde{\mathbf{d}}(0)e^{-\mathbf{L}t}$.



\subsection{Convergence of the Augmented Tracking Error}Finally, we analyze the impact of the observer convergence on the global tracking performance. Substituting the observer-augmented control law into the system dynamics yields the non-homogeneous error equation:
\begin{equation}\ddot{\mathbf{e}} + \mathbf{K}_d \dot{\mathbf{e}} + \mathbf{K}_p \mathbf{e} = \mathbf{M}^{-1}(q)\tilde{\mathbf{d}} \nonumber
\end{equation}
As $t \to \infty$, the NDOB ensures that $\tilde{\mathbf{d}} \to \mathbf{0}$ at a rate determined by $\mathbf{L}$. As the right-hand side of the equation asymptotically vanishes, the system reduces to a purely homogeneous linear ODE:\begin{equation}\ddot{\mathbf{e}} + \mathbf{K}_d \dot{\mathbf{e}} + \mathbf{K}_p \mathbf{e} = \mathbf{0}\end{equation}Since the gain matrices $\mathbf{K}_p$ and $\mathbf{K}_d$ are designed such that the characteristic polynomial is Hurwitz, the tracking error $\mathbf{e}$ is guaranteed to converge exponentially to zero. This confirms that the NDOB-augmented FBL architecture effectively eliminates steady-state pointing errors induced by parametric uncertainties and nonlinear disturbances.


\section{State Estimation via Extended Kalman Filter}
To achieve microradian-level pointing, the controller requires high-fidelity estimates of the gimbal states, including unobservable disturbances and sensor biases. We implement an Extended Kalman Filter (EKF) to fuse asynchronous data from 20-bit encoders, rate gyros, and a Quadrant Photodiode (QPD).
%\begin{comment}

\subsection{State and Measurement Definitions}
The augmented state vector $\mathbf{x} \in \mathbb{R}^{10}$ is defined to capture the coupled dynamics and sensor non-idealities:
\begin{equation}
\mathbf{x} = [\theta_{az}, \dot{\theta}_{az}, b_{az}, \theta_{el}, \dot{\theta}_{el}, b_{el}, \phi, \dot{\phi}, d_{az}, d_{el}]^T \nonumber
\end{equation}
where $\theta$ and $\dot{\theta}$ denote gimbal angles and velocities, $b$ represents gyro biases, $\phi$ is the optical roll error, and $d$ accounts for lumped disturbance torques (e.g., cable drag and friction). The measurement vector $\mathbf{z} \in \mathbb{R}^6$ is comprised of:
\begin{equation}
\mathbf{z} = [\theta_{az,enc}, \theta_{el,enc}, \dot{\theta}_{az,gyro}, \dot{\theta}_{el,gyro}, NES_{x,qpd}, NES_{y,qpd}]^T \nonumber
\end{equation}

%\subsection{Discrete-Time Prediction Phase}
%The state is propagated forward using a high-fidelity nonlinear model $\mathbf{f}(\hat{\mathbf{x}}_{k-1}, \mathbf{u}_{k-1})$ derived from the gimbal's Lagrangian dynamics. Given the sampling interval $\Delta t$, the a priori estimate and its associated covariance are:
%\begin{equation}\hat{\mathbf{x}}_{k|k-1} = \hat{\mathbf{x}}_{k-1|k-1} + \int_{t_{k-1}}^{t_k} \mathbf{f}(\mathbf{x}(\tau), \mathbf{u}(\tau)) d\tau \nonumber
%\end{equation}
%\begin{equation}
%\mathbf{P}_{k|k-1} = \mathbf{F}_k \mathbf{P}_{k-1|k-1} \mathbf{F}_k^T + \mathbf{Q}\nonumber
%\end{equation}
%where $\mathbf{F}_k = \left. \frac{\partial \mathbf{f}}{\partial \mathbf{x}} \right|_{\hat{\mathbf{x}}_{k-1|k-1}}$ is the state transition Jacobian, often computed via central-difference approximations to capture the nonlinearities in the Coriolis and mass matrices.

%subsection{Measurement Correction Phase}
%The correction reconciles the prediction with the sensor observations. The innovation $\mathbf{y}_k$ and its covariance $\mathbf{S}_k$ are computed as:
%\begin{equation}
%\mathbf{y}_k = \mathbf{z}_k - \mathbf{h}(\hat{\mathbf{x}}_{k|k-1})\nonumber
%\end{equation}
%\begin{equation}
%\mathbf{S}_k = \mathbf{H}_k \mathbf{P}_{k|k-1} \mathbf{H}_k^T + \mathbf{R}\nonumber
%\end{equation}
%where $\mathbf{h}(\cdot)$ is the nonlinear measurement model and $\mathbf{H}_k = \left. \frac{\partial \mathbf{h}}{\partial \mathbf{x}} \right|_{\hat{\mathbf{x}}_{k|k-1}}$ is the measurement Jacobian. For the 20-bit encoders, $H$ is a direct mapping, whereas for the QPD, it represents the optical sensitivity $\partial NES / \partial \theta$.

%The optimal Kalman gain $\mathbf{K}_k$ adjudicates between the $M, C, G$ model fidelity and sensor noise floors:
%\begin{equation}
%\mathbf{K}_k = \mathbf{P}_{k|k-1} \mathbf{H}_k^T \mathbf{S}_k^{-1}\nonumber
%\end{equation}

%The state update and the numerically robust Joseph-form covariance update are given by:
%\begin{equation}
%\hat{\mathbf{x}}_{k|k} = \hat{\mathbf{x}}_{k|k-1} + \mathbf{K}_k \mathbf{y}_k\nonumber
%\end{equation}
%\begin{equation}
%\mathbf{P}_{k|k} = (\mathbf{I} - \mathbf{K}_k \mathbf{H}_k) \mathbf{P}_{k|k-1}(\mathbf{I} - \mathbf{K}_k \mathbf{H}_k)^T + \mathbf{K}_k \mathbf{R} \mathbf{K}_k^T \nonumber
%\end{equation}
%%%%%%%%%%%%%%%%%%%%%%%%%%%%%%%%%
%\usepackage{amsmath, amssymb}

% --- EKF Section Start ---
\subsection{Extended Kalman Filter Formulation}
The Extended Kalman Filter (EKF) linearizes the nonlinear system dynamics and measurement models about the current state estimate using Taylor series expansions. Consider a discrete-time nonlinear system:
\begin{equation}
	\mathbf{x}_k = \mathbf{f}(\mathbf{x}_{k-1}, \mathbf{u}_{k-1}) + \mathbf{w}_{k-1}\nonumber
\end{equation}
\begin{equation}
	\mathbf{z}_k = \mathbf{h}(\mathbf{x}_k) + \mathbf{v}_k\nonumber
\end{equation}
where $\mathbf{w}_k \sim \mathcal{N}(0, \mathbf{Q})$ and $\mathbf{v}_k \sim \mathcal{N}(0, \mathbf{R})$ represent additive zero-mean Gaussian process and measurement noise, respectively.

\subsection{Prediction (Time Update)}
The state and error covariance are projected forward in time:
\begin{align}
	\hat{\mathbf{x}}_{k|k-1} &= \mathbf{f}(\hat{\mathbf{x}}_{k-1|k-1}, \mathbf{u}_{k-1}) \\
	\mathbf{P}_{k|k-1} &= \mathbf{F}_{k-1} \mathbf{P}_{k-1|k-1} \mathbf{F}_{k-1}^T + \mathbf{Q}\nonumber
\end{align}
where $\mathbf{F}_{k-1}$ is the state transition Jacobian matrix:
\begin{equation}
	\mathbf{F}_{k-1} = \left. \frac{\partial \mathbf{f}}{\partial \mathbf{x}} \right|_{\hat{\mathbf{x}}_{k-1|k-1}}\nonumber
\end{equation}

\subsection{Correction (Measurement Update)}
The filter corrects the predicted estimate using the innovation between the actual measurement $\mathbf{z}_k$ and the predicted measurement:
\begin{align}
	\mathbf{y}_k &= \mathbf{z}_k - \mathbf{h}(\hat{\mathbf{x}}_{k|k-1}) \\
	\mathbf{S}_k &= \mathbf{H}_k \mathbf{P}_{k|k-1} \mathbf{H}_k^T + \mathbf{R} \\
	\mathbf{K}_k &= \mathbf{P}_{k|k-1} \mathbf{H}_k^T \mathbf{S}_k^{-1}\nonumber
\end{align}
where $\mathbf{H}_k$ is the measurement Jacobian matrix:
\begin{equation}
	\mathbf{H}_k = \left. \frac{\partial \mathbf{h}}{\partial \mathbf{x}} \right|_{\hat{\mathbf{x}}_{k|k-1}}\nonumber
\end{equation}

The posterior state estimate and covariance are updated as follows:
\begin{align}
	\hat{\mathbf{x}}_{k|k} &= \hat{\mathbf{x}}_{k|k-1} + \mathbf{K}_k \mathbf{y}_k \\
	\mathbf{P}_{k|k} &= (\mathbf{I} - \mathbf{K}_k \mathbf{H}_k) \mathbf{P}_{k|k-1}\nonumber
\end{align}
% --- EKF Section End ---
%%%%%%%%%%%%%%%%%%%%%%%%%%%%%%%%%%
\subsection{Stochastic Tuning and Physical Interpretation}
%Tuning the $\mathbf{Q}$ and $\mathbf{R}$ matrices is pivotal for low-frequency disturbance rejection.
%\begin{itemize}
%	\item \textbf{Measurement Noise ($\mathbf{R}$):} Modeled using the quantization variance of the 20-bit encoders ($\sigma^2 = \Delta^2 / 12$) and the signal-to-noise ratio of the QPD.
%\item \textbf{Process Noise ($\mathbf{Q}$):} Higher values are assigned to the disturbance states $d_{az}, d_{el}$ to allow the filter to track non-periodic torque variations (e.g., mast vibrations) while maintaining a stiff model for known inertia.
%\end{itemize}


In state estimation for high-precision pointing, the statistical tuning of the EKF is not merely a numerical exercise but a physical mapping of system uncertainties. In the context of a 2-DOF gimbal subject to microradian constraints, the matrices $\mathbf{R}$, $\mathbf{Q}$, $\mathbf{P}$, and $\mathbf{S}$ serve as the fundamental mechanisms for balancing model-based prediction against stochastic sensor observations.



The measurement noise covariance ($\mathbf{R}$) matrix quantifies the "noise floor" of the sensor suite. In this implementation, the diagonal elements for the 20-bit encoders are determined by their quantization limit, where $\sigma_{enc}^2 = \frac{\Delta^2}{12}$. However, for the Rate Gyros, $\mathbf{R}$ must account for both thermo-mechanical white noise and high-frequency vibrations.A critical design consideration here is that a lower $\mathbf{R}$ value forces the filter to "trust" the sensors more, providing rapid response to pointing deviations. However, if $\mathbf{R}$ is tuned too aggressively low, the estimator may propagate high-frequency sensor noise into the control loop, causing "jitter" that degrades the microradian stability of the laser link.




While $\mathbf{R}$ is often static, the   process noise covariance $\mathbf{Q}$ matrix is the primary tool for handling Model-Plant Mismatch. It represents the uncertainty in the Lagrangian dynamics (the $M, C, G$ matrices).  In this research, we employ a Kinematic Noise Model. We assign significantly higher process noise to the disturbance states ($d_{az}, d_{el}$) compared to the position states. This allows the EKF to treat unmodeled cable torques and mast vibrations as "stochastic walks." By "un-stiffening" the disturbance states through $\mathbf{Q}$, the filter can rapidly estimate the residual torque, which is then fed into the NDOB for cancellation.



The innovation covariance  matrix, $\mathbf{S}_k$ %= \mathbf{H}_k \mathbf{P}_{k|k-1} \mathbf{H}_k^T + \mathbf{R}$ 
represents the total projected uncertainty in the measurement space.  During steady-state tracking, $\mathbf{S}$ is dominated by the sensor noise floor $\mathbf{R}$. During high-acceleration slews or transient disturbances, the $\mathbf{H}\mathbf{P}\mathbf{H}^T$ term expands, reflecting the growth in model uncertainty.  In a production-grade EKF, $\mathbf{S}$ is used for Innovation Gate-keeping. If the normalized innovation $\mathbf{y}_k^T \mathbf{S}_k^{-1} \mathbf{y}_k$ exceeds a 3-sigma threshold, it indicates a sensor malfunction or a mechanical collision, allowing the safety logic to inhibit the update.



The state error covariance ($\mathbf{P}$) is the filter’s self-assessment of its estimation accuracy. In the context of laser communication, the diagonal elements $\mathbf{P}_{ii}$ corresponding to $\theta_{az}$ and $\theta_{el}$ provide the Pointing Confidence Interval.Unlike $\mathbf{Q}$ and $\mathbf{R}$, which are user-defined, $\mathbf{P}$ evolves via the Riccati equation. A key observation in this study is the Covariance Collapse phenomenon. If the gimbal remains stationary for too long with high-quality sensors, $\mathbf{P}$ may become artificially small, causing the filter to become "blind" to new disturbances. To prevent this, we maintain a lower bound on the diagonal of $\mathbf{Q}$ to ensure the filter remains perpetually "alert" to changing environment torques.

\subsection{ Design Guidelines:Balancing Estimator Bandwidth and Observer Stability} 

The integration of an EKF with an NDOB introduces a fundamental trade-off between noise attenuation and disturbance rejection. For microradian-level pointing, the designer must reconcile the spectral properties of the estimator with the passband of the observer.


\subsubsection{The Bandwidth Separation Principle:} The EKF bandwidth is primarily governed by the ratio of $\mathbf{Q}$ to $\mathbf{R}$. A "fast" EKF (high $\mathbf{Q}$) provides low-latency state estimates but propagates sensor noise into the control loop. Conversely, the NDOB bandwidth is determined by the observer gain $L$. To ensure closed-loop stability, we propose the following heuristic:\begin{equation}\omega_{control} < \omega_{NDOB} < \omega_{EKF}\end{equation}The EKF must converge faster than the NDOB to ensure that the disturbance estimate $\hat{d}$ is based on a reliable velocity state. If the NDOB attempt to cancel disturbances based on a lagging or "slow" EKF estimate, it will introduce a phase lag that can lead to limit-cycle oscillations at the gimbal's resonant frequencies.

\subsubsection{Sensitivity to Sensor Noise ($\mathbf{R}$) vs. Observer Gain ($L$):} In precision optical tracking, high-frequency "jitter" is often more detrimental than slow-varying offsets.

\begin{itemize}
\item  Case 1: Aggressive NDOB ($L$ is high). This provides near-perfect rejection of low-frequency friction and cable torque. However, it sensitive to the "noise-aliasing" effect where high-frequency innovation from the EKF is interpreted by the NDOB as a torque disturbance, causing the motor to "chatter."

\item Case 2: Conservative EKF (High $\mathbf{R}$). This smooths the encoder quantization noise but delays the detection of sudden mast vibrations.
\end{itemize}



%\begin{itemize}
In this research we utilized the EKF's Innovation Covariance $\mathbf{S}_k$ as a dynamic scaling factor for the NDOB gain. When $\mathbf{S}_k$ is large (indicating transient uncertainty), the NDOB gain $L$ should be temporarily attenuated to prevent the injection of erroneous torque commands.

\subsubsection{Mitigation of Peaking Phenomena} Nonlinear systems are susceptible to "peaking"—high-amplitude transients during the observer’s convergence phase. In a 2-DOF gimbal, peaking in the Azimuth channel can couple into the Elevation channel through the mass matrix $\mathbf{M}(\mathbf{q})$.

In the simulation we implemented a "Cold-Start" saturation limit on the NDOB output. During the initial acquisition phase (where $\mathbf{P}_0$ is high), the NDOB correction signal is clipped to $20\%$ of the maximum motor torque until the EKF residuals $\mathbf{y}_k$ settle within the 1-sigma bound.
\subsubsection{Parameter Sensitivity and Robustness}
While the FBL controller relies on the nominal model $\mathbf{M}$ and $\mathbf{G}$, the NDOB is designed to absorb the residuals. However, if the inertia estimate used in the EKF is significantly different from the true plant, the EKF will produce a biased velocity estimate.
%\begin{itemize}
Therefore, we performed a sensitivity analysis on the $\mathbf{Q}$ matrix. For systems with high parameter uncertainty, the $Q$ values for the velocity states should be increased relative to the position states. This "loosens" the filter’s reliance on the flawed physics model and forces it to follow the 20-bit encoder data more strictly, providing a cleaner residual for the NDOB to process.
% \end{itemize}
%\end{itemize}

%
%\begin{equation}
%\label{deqn_ex1a}
%x = \sum_{i=0}^{n} 2{i} Q.
%\end{equation}


%\begin{figure}[!t] %helps with placement (here, top, bottom, page)
%\centering
%\includegraphics[width=2.5in]{Figure_1.png} 
%\caption{A single column figure with six graphics.}
%\label{FigGam}
%\end{figure}
% --- EKF Equations for Gimbal State Estimation ---


\section{Environmental Disturbance and Wind Loading Models}
\label{sec:disturbances}

To evaluate the robustness of the proposed NDOB-augmented feedback linearization controller under realistic operating conditions, a high-fidelity environmental disturbance suite is incorporated into the simulation framework. These disturbances are injected exclusively into the plant dynamics, thereby intentionally introducing plant--model mismatch. The disturbed gimbal dynamics are described as:
\begin{equation}
\mathbf{M}(\mathbf{q})\ddot{\mathbf{q}} + \mathbf{C}(\mathbf{q}, \dot{\mathbf{q}})\dot{\mathbf{q}} + \mathbf{G}(\mathbf{q})
= \boldsymbol{\tau}_{\mathrm{ctrl}} + \boldsymbol{\tau}_{\mathrm{dist}} \nonumber
\end{equation}
where $\boldsymbol{\tau}_{\mathrm{dist}}$ denotes the lumped exogenous torque vector arising from aerodynamic loading and structural vibrations.

\subsection{Aerodynamic Loading: Dryden Turbulence Model}

For Free-Space Optical (FSO) terminals, wind-induced aerodynamic loading constitutes a dominant source of pointing jitter and tracking degradation. The wind-induced torque acting on the gimbal assembly is:
\begin{equation}
\tau_{\mathrm{wind}}(t) = \frac{1}{2} \rho C_d A L \, V^2(t) \nonumber
\end{equation}
where $\rho$ is the air density, $C_d$ is the drag coefficient, $A$ is the effective projected area exposed to the wind, and $L$ is the moment arm relative to the gimbal rotation axis. The instantaneous wind velocity is decomposed as:
\begin{equation}
V(t) = V_{\mathrm{mean}} + V_{\mathrm{gust}}(t) \nonumber
\end{equation}
where $V_{\mathrm{mean}}$ denotes the steady mean wind speed and $V_{\mathrm{gust}}(t)$ represents the stochastic gust component.



The gust velocity $V_{\mathrm{gust}}(t)$ is generated using the longitudinal Dryden turbulence model \cite{beal1993digital}, which characterizes atmospheric turbulence as the output of a shaping filter driven by zero-mean Gaussian white noise. The corresponding continuous-time transfer function is given by:
\begin{equation}
H_u(s) = \sigma_u \sqrt{\frac{2 V_{\mathrm{mean}}}{\pi L_u}} \frac{1 + \sqrt{3}\frac{L_u}{V_{\mathrm{mean}}} s}{\left(1 + \frac{L_u}{V_{\mathrm{mean}}} s\right)^2} \nonumber
\end{equation}
where $L_u$ is the longitudinal turbulence scale length and $\sigma_u$ denotes the turbulence intensity. For low-altitude, mast-mounted FSO terminals, $L_u$ is typically selected in the range of $100$--$300$~m. This formulation effectively captures both the low-frequency bias associated with steady wind and the high-frequency spectral content of gusts, providing a stringent excitation for evaluating the estimation bandwidth of the NDOB \cite{huang2018high}.

\subsection{Structural Vibration and Base Motion}

In addition to aerodynamic disturbances, platform-induced jitter resulting from mast flexibility, vehicle motion, or mechanical harmonics is modeled as structural vibration at the gimbal base. These disturbances are synthesized using a power spectral density (PSD) representation composed of multiple lightly damped structural modes\cite{gawronski2004advanced}:
\begin{equation}
S_a(f) = \sum_{i=1}^{N} \frac{A_i}{\left(1 - \left(\frac{f}{f_i}\right)^2\right)^2 + \left(2 \zeta_i \frac{f}{f_i}\right)^2} \nonumber
\end{equation}
where $f_i$, $\zeta_i$, and $A_i$ denote the natural frequency, damping ratio, and modal amplitude of the $i$-th vibration mode, respectively.

The resulting translational base acceleration $a(t)$ and angular acceleration $\alpha(t)$ are mapped into equivalent disturbance torques through the inertial properties of the gimbal.
%\begin{equation}
%\tau_{\mathrm{vib}}(t) = J \alpha(t) + m l a(t),
%\end{equation}
%where $J$ is the rotational inertia of the gimbal assembly and $m$ is the effective mass contributing to translational excitation.


\begin{table}[!t]
\caption{Simulation Configuration for Environmental Disturbances\label{tab:disturbance_params}}
\centering
\begin{tabular}{l c r}
\hline
Parameter & Value & Units/Description \\
\hline
\textbf{Wind Model (Dryden)} & & \\
Mean Velocity ($V_{\text{mean}}$) & 8.0 & m/s \\
Turbulence Intensity ($\sigma_u/V_{\text{mean}}$) & 0.15 & Moderate conditions \\
Scale Length ($L_u$) & 200.0 & m (MIL-F-8785C) \\
Wind Direction & 45.0 & Degrees (Cross-axis) \\
Onset Time ($t_{\text{start, wind}}$) & 2.0 & s \\
\hline
\textbf{Structural Vibration} & & \\
Modal Frequencies ($f_i$) & [15.0, 45.0, 80.0] & Hz \\
Damping Ratios ($\zeta_i$) & [0.02, 0.015, 0.01] & Lightly damped \\
PSD Amplitudes ($A_i$) & [1e-3, 5e-4, 2e-4] & $(m/s^2)^2/Hz$ \\
Inertia Coupling & 0.1 & $N \cdot m / (m/s^2)$ \\
Onset Time ($t_{\text{start, vib}}$) & 4.0 & s \\
\hline
\textbf{Structural Noise} & & \\
Noise Std. Dev. ($\sigma_{\text{noise}}$) & 0.01 & $N \cdot m$ \\
Passband Range & 100.0 -- 500.0 & Hz \\
\hline
Seed (Reproducibility) & 42 & Integer \\
\hline
\end{tabular}
\end{table}

\section{Simulation Results and Performance Analysis}To evaluate the efficacy of the proposed control architecture, a high-fidelity numerical simulation was developed using Python. The study provides a comparative performance analysis among three distinct control strategies: a conventional Proportional-Integral-Derivative (PID) controller, a standard Feedback Linearization (FBL) controller, and the proposed Observer-Augmented Feedback Linearization (FBL+NDOB) scheme.

\subsection{ Simulation Setup and Environmental Perturbations}All controllers were synthesized based on the nominal model parameters detailed in Table I. To rigorously test the robustness of the system, a significant model-plant mismatch was intentionally introduced in the simulation environment. Specifically, the "actual" plant (the simulation model) incorporates unmodeled dynamics, nonlinear friction, and parametric uncertainties. These uncertainties represent common industrial challenges, such as inaccurate moment-of-inertia calculations for gimbal assemblies and center-of-gravity (CoG) offsets relative to the elevation axis.

\subsection{Tracking Performance in Extreme Kinematic Scenarios}
The tracking efficacy of the proposed architecture is evaluated using a dual-axis sinusoidal trajectory, as depicted in Fig.~\ref{Fig:tracking}. The simulation involves a dynamic target acquisition phase where the laser source follows a periodic path across both axes. A critical stationary dwell is maintained at an elevation of $45^{\circ}$ to observe steady-state characteristics. 

Furthermore, a comprehensive sweep is conducted across the full range of both azimuth and elevation axes, specifically targeting a peak elevation of $90^{\circ}$. This "zenith-pointing" condition represents the most demanding kinematic scenario for a two-axis gimbal. At this orientation, the system approaches a mathematical singularity, which maximizes nonlinear cross-axis coupling and amplifies the sensitivity to modeling uncertainties. 



The primary performance metric is the \textit{Pointing Error}, defined as the instantaneous angular divergence between the incoming laser Line-of-Sight (LoS) vector and the gimbal pointing vector (see Fig.~\ref{Fig:error}). This error is derived via a rigorous sequence of coordinate transformations—mapping the ground-fixed inertial frame through the optical and sensor-fixed frames—ensuring that the reported precision accounts for all kinematic non-idealities.

\subsection{Comparative Analysis of Control Strategies}
As illustrated in Fig.~\ref{Fig:tracking}, both the PID and the NDOB-augmented FBL controllers successfully achieve trajectory tracking; however, their respective precision regimes differ substantially. Fig.~\ref{Fig:error} presents the angular tracking error as a temporal function, revealing distinct performance bifurcations.

The standalone FBL controller exhibits significant performance degradation as the elevation angle increases. While FBL is theoretically sound, it lacks the inherent robustness required to compensate for unaccounted disturbance torques arising from inertia mismatches and nonlinear friction. Conversely, the PID controller—while functional during nominal tracking—demonstrates a steady increase in error as the system moves away from its $0^{\circ}$ design equilibrium.



In contrast, the FBL+NDOB architecture maintains superior pointing accuracy, constraining the error below $0.1^{\circ}$ throughout the entire sweep. By treating modeling uncertainties and external perturbations as lumped disturbance torques, the NDOB effectively estimates and cancels the nonlinear residuals that the FBL alone cannot mitigate. 

Crucially, during the $90^{\circ}$ sweep, the performance of both the PID and standalone FBL controllers significantly degrades. The PID controller, typically tuned for small-angle linearized dynamics, suffers from gain-mismatch at high elevations. While gain-scheduling could potentially mitigate this, as noted in previous literature, the proposed FBL+NDOB architecture achieves superior results without the complexity of scheduling, maintaining sub-$0.1^{\circ}$ precision even during the most stringent kinematic phases.
\begin{table}[!t]
\caption{Gimbal Parameters (ideal case)\label{tab:table1}}
\centering
\begin{tabular}{l c r}
\hline
Symbol & Description & Value\\
\hline
$\mathbf{J}_{az}$ & inertia of azimuth assembly & 0.00038 $Kg.m^2$\\
$\mathbf{J}_{el}$ & inertia of elevation assembly & 0.00018 $Kg.m^2$\\

$\mathbf{m}_{az}$ & mass of azimuth assembly & 1 $Kg$\\

$\mathbf{m}_{el}$ & mass of elevation assembly  & 0.5 $Kg$\\
$c $ &  c.g along x-axis  &  $0m$\\
$h $ &  c.g along y-axis  &  $0m$\\

\hline
\end{tabular}
\end{table}

\begin{table}[!b]
\caption{Gimbal Parameters (realistic)\label{tab:table2}}
\centering
\begin{tabular}{l c r}
\hline
Symbol & Description & Value\\
\hline
$\mathbf{J}_{az}$ & inertia of azimuth assembly &  $0.00038 \pm 15\%  Kg.m^2$\\
$\mathbf{J}_{el}$ & inertia of elevation Assembly & $ 0.00018  \pm 15\% Kg.m^2$\\
$\mathbf{m}_{az}$ & mass of azimuth assembly & 1 $Kg$\\
$\mathbf{m}_el$ & mass of elevation assembly  & 0.5 $Kg$\\
$c $ &  c.g along x-axis  &  $0.0002m$\\
$h $ &  c.g along y-axis  &  $0.0003m$\\

\hline
\end{tabular}
\end{table}

%\section{Results and Analysis}


%\begin{figure}[!t]
%\centering
%\includegraphics[width=2.5in]{fig1}
%\caption{Simulation results for the network.}
%	\label{fig_1}
%\end{figure}



%\begin{comment}
\begin{figure}[!t] % [htbp] helps with placement (here, top, bottom, page)
\centering
%\includegraphics[width=3in]{fig1_position_tracking.png} 
\caption{Gimbal trajectory tracking performance for the proposed and baseline control strategies.}
\label{Fig:tracking}
\end{figure}

\begin{figure}[!t] % [htbp] helps with placement (here, top, bottom, page)
\centering
%\includegraphics[width=3in]{fig2_tracking_error_handover.png} 
\caption{Comparative tracking performance in terms of LOS error under both nominal and disturbed operating conditions.}
\label{Fig:error}
\end{figure}

\begin{figure}[!t] % [htbp] helps with placement (here, top, bottom, page)
\centering
%\includegraphics[width=3in]{fig7_performance_summary.png} 
\caption{Performance summary across all control architectures.}
\label{Fig:performace}
\end{figure}
\begin{figure*}[!t] % [htbp] helps with placement (here, top, bottom, page)
\centering
%\includegraphics[width=6in]{fig12_environmental_disturbances.png} 
\caption{Temporal signals of the introduced atmospheric disturbances and platform-induced structural vibrations.}
\label{Fig:disturbance}
\end{figure*}
\begin{figure}[!t] % [htbp] helps with placement (here, top, bottom, page)
\centering
%\includegraphics[width=3in]{fig9_fsm_performance.png} 
\caption{FSM compensatory motion in response to wind loading and structural vibrations.}
\label{Fig:fsm}
\end{figure}



\section{ Fine-Pointing Stage: Fast Steering Mirror (FSM) Integration}While the NDOB-augmented FBL control ensures high-accuracy tracking of the coarse gimbal, residual mechanical jitter and high-frequency structural resonances often exceed the torque-loop bandwidth of the gimbal actuators. To mitigate these effects, a Fast Steering Mirror (FSM) is employed as a secondary, high-bandwidth compensation stage in such  systems \cite{kim2024system,li2024system}.




\subsection{Robustness to External Disturbance Ingress}To evaluate the resilience of the dual-stage system, a stochastic noise perturbation was injected into the simulation at $t = 5$ s. This perturbation represents an aggressive external disturbance, such as high-frequency platform vibrations or atmospheric scintillation effects.As illustrated in the performance Fig.~\ref{Fig:disturbance} , the system behavior is analyzed in two distinct regimes:\begin{enumerate}
\item Quiescent Tracking ($t < 5$ s): The FSM  maintains a near-zero steady-state error, nullifying the minute residuals left by the NDOB coarse loop as shown in Fig.~\ref{Fig:fsm}. During this phase, the pointing jitter is confined to the sub-microradian regime, demonstrating the high fidelity of the EKF state estimation.The pure FBL control completely looses sight as pointing angle approaches 45 degrees along both the axis. During this phase both PID and the  FBL+NDOB  show comparable performance as depicted in Fig.~\ref{Fig:error}.

\item Disturbance Rejection ($t \geq 5$ s): Immediately following the introduction of external noise at the 5-second mark, a transient spike in the pointing error is observed as shown in Fig.~\ref{Fig:error}. However, the coarse  control loop (FBL+NDOB) demonstrates exceptional disturbance rejection, suppressing the ingress and returning the tracking error to its baseline within milliseconds. 
\end{enumerate}

\subsection{Synergistic Coarse-Fine Interaction}The simulation results confirm that the FSM prevents high-frequency "noise ingress" from saturating the gimbal's slower actuators. Conversely, the NDOB in the coarse loop provides a cleaner operational base for the FSM by removing low-frequency friction "stiction" and nonlinear coupling. Without the NDOB, the coarse gimbal would exhibit non-linear jumps that could exceed the limited angular stroke of the FSM's flexures. This synergistic relationship ensures that the composite pointing error remains within the stringent requirements of the optical budget even under adverse conditions.

\section{ Conclusion and Future Work}

\subsection{Conclusion}This paper presented a multi-layered control and estimation architecture designed to achieve microradian-level pointing accuracy for 2-DOF gimbal systems in the presence of nonlinearities and stochastic disturbances. The core of the proposed solution lies in the synergy between three distinct yet integrated stages:

\begin{enumerate}
	\item State Estimation: The implementation of an Extended Kalman Filter (EKF) proved essential for sensor fusion, successfully reconciling asynchronous data from 20-bit encoders and rate gyros. By modeling gyro biases and optical roll error within the 10-element state vector, the EKF provided the high-fidelity velocity estimates required for subsequent control stages.
	
	\item Robust Coarse Control: The NDOB-augmented Feedback Linearization (FBL) architecture effectively addressed the "reality gap" between the nominal $M, C, G$ dynamics and the actual plant. Simulation results demonstrated that while pure FBL fails under parametric uncertainty and friction, the NDOB successfully estimates lumped disturbances, reducing steady-state tracking error by an order of magnitude, particularly during high-elevation slews.
	
	\item Active Jitter Rejection: The integration of a Fast Steering Mirror (FSM) as a fine-pointing stage provided the necessary bandwidth to reject high-frequency noise ingress. The disturbance injection tests at $t = 5\,\text{s}$ confirmed that the dual-stage architecture maintains link stability even under aggressive external perturbations that would otherwise saturate a standalone gimbal loop.
\end{enumerate}

In summary, the results validate that a hierarchical approach—coupling nonlinear robust control with high-bandwidth optical compensation—is a viable path for the next generation of industrial laser communication terminals.

\subsection{Future Work}While the current architecture demonstrates significant robustness, several avenues for future research remain:\begin{itemize}
	\item Adaptive NDOB Gains: Future iterations could explore the use of fuzzy logic or neural networks to adaptively tune the observer gain matrix $\mathbf{L}$ in real-time, optimizing the trade-off between disturbance rejection and noise amplification.
	\item Structural Resonance Filtering: Incorporating notch filters or Input Shaping techniques into the FBL outer-loop could further mitigate the excitation of high-frequency structural modes during rapid slewing maneuvers.
	
	\item Experimental Validation: The authors intend to transition this "digital twin" simulation to a physical testbed to investigate the impact of communication latency and computational overhead on the EKF-NDOB-FSM synchronization.
\end{itemize}
%% Use \section commands to start a section

%% If you have bib database file and want bibtex to generate the
%% bibitems, please use
%%
%%  \bibliographystyle{elsarticle-num} 
%%  \bibliography{<your bibdatabase>}

%% else use the following coding to input the bibitems directly in the
%% TeX file.

%% Refer following link for more details about bibliography and citations.
%% https://en.wikibooks.org/wiki/LaTeX/Bibliography_Management

\begin{thebibliography}{00}

\bibliographystyle{IEEEtran}
\bibitem{tran2024observer}Tran, X., Nguyen, V., Le, P. \& Kang, H. Observer-based fault-tolerant control for uncertain robot manipulators without velocity measurements. {\em Actuators}. \textbf{13}, 207 (2024)
\bibitem{spong2006robot}Spong, M., Hutchinson, S., Vidyasagar, M. \& Others Robot modeling and control. (Wiley New York,2006)
\bibitem{li2024system}Li, Z., Li, L., Zhang, J. \& Feng, W. System modeling and sliding mode control of fast steering mirror for space laser communication. {\em Mechanical Systems And Signal Processing}. \textbf{211} pp. 111206 (2024)
\bibitem{kim2024system}Kim, J. System integration design of high-performance piezo-actuated fast-steering mirror for laser beam steering system. {\em Sensors}. \textbf{24}, 6775 (2024)
\bibitem{dhruv2025review}Dhruv \& Kaushal, H. A Review of Pointing Modules and Gimbal Systems for Free-Space Optical Communication in Non-Terrestrial Platforms. {\em Photonics}. \textbf{12}, 1001 (2025)
\bibitem{dao2023nonlinear}Dao, H., Nguyen, M. \& Ahn, K. Nonlinear functional observer design for robot manipulators. {\em Mathematics}. \textbf{11}, 4033 (2023)
\bibitem{lv2023modeling}Lv, T., Ruan, P., Jiang, K. \& Jing, F. Modeling and analysis of fast steering mirror disturbance effects on the line of sight jitter for precision pointing and tracking system. {\em Mechanical Systems And Signal Processing}. \textbf{188} pp. 110002 (2023)
\bibitem{li2024system}Li, Z., Li, L., Zhang, J. \& Feng, W. System modeling and sliding mode control of fast steering mirror for space laser communication. {\em Mechanical Systems And Signal Processing}. \textbf{211} pp. 111206 (2024)
\bibitem{olsson1998friction}
H. Olsson, K. J. {\AA}str{\"o}m, C. Canudas de Wit, M. G{\"a}fvert, and P. Lischinsky, ``Friction models and friction compensation,'' \emph{Eur. J. Control}, vol. 4, no. 3, pp. 176--195, 1998.
\bibitem{chen2015disturbance}Chen, W., Yang, J., Guo, L. \& Li, S. Disturbance-observer-based control and related methods—An overview. {\em IEEE Transactions On Industrial Electronics}. \textbf{63}, 1083-1095 (2015)
\bibitem{cline2017design}Cline, A., Shubert, P., McNally, J., Jacka, N. \& Pierson, R. Design of a stabilized, compact gimbal for space-based free space optical communications (FSOC). {\em Free-Space Laser Communication And Atmospheric Propagation XXIX}. \textbf{10096} pp. 113-123 (2017)
\bibitem{naderolasli2020two}
A. Naderolasli and M. Tabatabaei, ``Two-axis gimbal system stabilization using adaptive feedback linearization,'' \emph{Recent Advances in Electrical \& Electronic Engineering}, vol. 13, no. 3, pp. 355--368, 2020.
\bibitem{9279120}Wei, L. \& Chen, G. Extended High-gain Observer Based Output Feedback Linearization of Robot Manipulator. {\em 2020 10th Institute Of Electrical And Electronics Engineers International Conference On Cyber Technology In Automation, Control, And Intelligent Systems (CYBER)}. pp. 73-79 (2020)
\bibitem{xu2022hybrid}Xu, D., Hu, T., Ma, Y. \& Shu, X. A Hybrid State/Disturbance Observer-Based Feedback Control of Robot with Multiple Constraints. {\em Sensors}. \textbf{22}, 9112 (2022)
\bibitem{aboudonia2016disturbance}Aboudonia, A., El-Badawy, A. \& Rashad, R. Disturbance observer-based feedback linearization control of an unmanned quadrotor helicopter. {\em Proceedings Of The Institution Of Mechanical Engineers, Part I: Journal Of Systems And Control Engineering}. \textbf{230}, 877-891 (2016)
\bibitem{masten2008inertially}Masten, M. Inertially stabilized platforms for optical imaging systems. {\em IEEE Control Systems Magazine}. \textbf{28}, 47-64 (2008)
\bibitem{chen2015disturbance}Chen, W., Yang, J., Guo, L. \& Li, S. Disturbance-observer-based control and related methods—An overview. {\em IEEE Transactions On Industrial Electronics}. \textbf{63}, 1083-1095 (2015)
\bibitem{huang2018high}Huang, L., Wu, Z. \& Wang, K. High-precision anti-disturbance gimbal servo control for control moment gyroscopes via an extended harmonic disturbance observer. {\em IEEE Access}. \textbf{6} pp. 66336-66349 (2018)
\bibitem{milf8785c}Department of Defense Military Specification: Flying Qualities of Piloted Airplanes.  (1980)
\bibitem{beal1993digital}Beal, T. Digital simulation of atmospheric turbulence for Dryden and von Karman models. {\em Journal Of Guidance, Control, And Dynamics}. \textbf{16}, 132-138 (1993)
\bibitem{gawronski2004advanced}Gawronski, W. Advanced structural dynamics and active control of structures. (Springer,2004)

\end{thebibliography}
\end{document}

\endinput
%%
%% End of file `elsarticle-template-num.tex'.
